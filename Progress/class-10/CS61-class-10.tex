\documentclass[a4paper, 11pt]{article}

%%%%%% use packages %%%%%%
\usepackage{CJKutf8}
\usepackage{graphicx}
\usepackage[unicode,pdftex]{hyperref}
\usepackage{indentfirst}
\usepackage{latexsym}
\usepackage{amsmath}
\usepackage{flafter}
\usepackage{enumerate}
\usepackage{leftidx}

%%%%%% indent settings %%%%%%
\setlength{\parindent}{20pt}
\setlength{\parskip}{1ex plus 0.5ex minus 0.25ex}
\setlength{\baselineskip}{1.0ex}
\renewcommand{\baselinestretch}{1.5}

\begin{CJK}{UTF8}{gbsn}

\title{CS61第十次课程记录}
\author{傅海平\\
\textsc{Institute Of Computing Technology,}\\
\textsc{Chinese Academy Of Sciences}\\
haipingf@gmail.com\\
}
\date{\today}
\begin{document}
\maketitle
\newpage
\tableofcontents
\newpage
\section{Topics}
\begin{center}
  \Large{线程}
\end{center}

\section{progress}
早上8:30点开始,8:30 - 10:30 学习 lec18.pdf 课程讲义,然后11:00开始讨论学习过程中遇到的问题。
\section{learning details}
\subsection{course sketch}
\subsubsection{Why to synchronize multiple threads}
\begin{itemize}
  \item{Interleaved Execution}
  \item{多个线程执行可以抢占}
  \item{顺序一致性}
  \item{多处理器\&多核情况下调度,Cache 管理}
  \item{Relaxed Memory Models(Sequential-consistent ordering \& Relaxed
	ordering \& acquire-release ordering)}
	\begin{itemize}
	  \item{Seqential Consistent Ordering}
		\begin{itemize}
		  \item{sequential-consistent model 是沿袭 Java 的内存模型(确切的说是
			sequential-consistent with data-race-free),所有的 atomic 操作都可以
			看作满足唯一 total order 的操作,也就是说,可以把这样的多线程程序理
			解成一个有先有后交叉运行的单线程程序(对atomic type而言),这在推理时
			是有帮助的。每个 shared atomic 对象的改变在每个线程看来都是一样的,
			包括时间顺序以及值。}
		  \end{itemize}
	  \end{itemize}

	\begin{itemize}
	  \item{Relaxed Memory Ordering}
		\begin{itemize}
		  \item{完全不参与多线程间的同步,编译器可以随便优化。}
		  \item{对于自身线程,data-dependency 产生的依赖需要满足,这是单线程程序
			正确性的要求。}
		  \item{此外它唯一的作用就是属于原子操作,所以不会产生 data race。}
			
		  \end{itemize}
	  \end{itemize}

	\begin{itemize}
	  \item{Acquire Release Ordering}
		\begin{itemize}
		  \item{该语义即同步语义,每一个 acquire 对应一个 release(跨线程的),详
			细定义参见标准(29.3-2)。需要注意的是 C++ 的同步语义不是先验的,我们
			无法通过之前的执行判断之后两个操作是否同步了,而只能通过推理所有的情
			况来验证程序的正确性。具体来说,对于一个可能的 acquire-release pair
			,我们必须考虑他们同步时的情况,也需要考虑他们没有同步时的情况。推理
			C++ 多线程程序的正确性,特别是在这两种 ordering 情况下,需要塑模
			happens-before 的关系,通俗点说,就是推理出相关操作之间
			所有可能发生的顺序(或者根本就无序),所以说带锁的多线程的程序很难写正
			确,更别说无锁的,可见一斑。}
		  \end{itemize}
	  \end{itemize}
  \end{itemize}
\subsubsection{Race conditions}
\begin{itemize}
  \item{并行程序访问共享变量时没有线程\&进程间的同步。}
  \item{程序执行结果是非确定的。}
  \item{现实中的例子}
  \item{解决方案}
  \end{itemize}
\subsubsection{Mutual exclusion and critical sections}
\begin{itemize}
  \item{互斥和临界区}
  \item{临界区定义}
  \end{itemize}
\subsubsection{Locks}
\begin{itemize}
  \item{acquire}
  \item{release}
  \item{锁机制实现}
	\begin{itemize}
	  \item{实现锁机制需要硬件支持}
	  \item{禁止中断:防止上下文切换,但是该方法只在单处理器上有效}
	  \end{itemize}
  \end{itemize}

\begin{itemize}
  \item{spinlock自旋锁}
  \item{原子操作:CPU保证整个操作是原子的}
	\begin{itemize}
	  \item{Test-and-set}
	  \item{Compare-and-swap}
	  \end{itemize}
  \item{忙等待,效率}
  \item{用户空间锁机制实现:futex}
  \end{itemize}
\subsubsection{Efficiently implementing locks}
\begin{itemize}
  \item{见讲义}
  \end{itemize}
\subsection{Problems}
\subsection{Solutions}
\section{测试}
\newpage
\end{CJK}
\end{document}
