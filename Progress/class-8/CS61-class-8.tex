\documentclass[a4paper, 11pt]{article}

%%%%%% use packages %%%%%%
\usepackage{CJKutf8}
\usepackage{graphicx}
\usepackage[unicode]{hyperref}
\usepackage{indentfirst}
\usepackage{latexsym}
\usepackage{amsmath}
\usepackage{flafter}
\usepackage{enumerate}
\usepackage{leftidx}

%%%%%% indent settings %%%%%%
\setlength{\parindent}{20pt}
\setlength{\parskip}{1ex plus 0.5ex minus 0.25ex}
\setlength{\baselineskip}{1.0ex}
\renewcommand{\baselinestretch}{1.5}

\begin{CJK}{UTF8}{gbsn}


\title{CS61第八次课程记录}
\author{傅海平\\
\textsc{Institute Of Computing Technology,}\\
\textsc{Chinese Academy Of Sciences}\\
haipingf@gmail.com\\
}
\date{\today}
\begin{document}
\maketitle
\newpage
\tableofcontents
\newpage
\section{Topics}
\begin{center}
  \Large{程序的链接和加载 \& 线程}
\end{center}

\section{progress}
早上8:30点开始,8:30 - 10:50 学习 lec16-linking\_loading.pdf 和
lec17-threads.pdf 两张课程讲义,然后11:00开始讨论学习过程中遇到的问题
。
\section{learning details}
\subsection{course sketch}
\subsubsection{the what and why of linking}
\begin{itemize}
  \item{链接}
	\begin{itemize}
	  \item{链接定义}
	  \item{为什么需要链接?}
	  \item{链接在什么阶段完成?}
	  \end{itemize}
  \item{静态链接}
	  \begin{itemize}
		\item{编译器完成翻译和链接:gcc 是编译器的驱动程序,它会调用 c 预处理器
		  ,c 编译器,汇编器和链接编辑器}
		\end{itemize}
   \item{链接器}
	 \begin{itemize}
	   \item{定义}
	   \item{3个基本任务:拷贝数据和代码到可执行文件,解析符号,重定位符号(绝
		 对地址,而非相对地址)}
	   \item{为什么需要连接器?}
		 \begin{itemize}
		   \item{模块化}
		   \item{效率}
		   \end{itemize}
	   \end{itemize}
\end{itemize}
\subsubsection{symbol relocation}
\begin{itemize}
  \item{链接操作}
  \end{itemize}
\subsubsection{symbol resolution}
\begin{itemize}
  \item{符号解析}
  \item{strong \& weak symbols}
	\begin{itemize}
	\item{strong symbols:函数名,初始化的全局变量}
	\item{weak symbols:未初始化的全局变量}
	\end{itemize}
  \item{链接规则}
	\begin{itemize}
	  \item{多个同名的强符号不能同时存在}
	  \item{给定同名的强符号和多个弱符号,连接器选择强符号}
	  \item{给定多个弱符号,则链接器任意选择一个}
	  \end{itemize}
	\item{static 关键字: 影响符号的生命期和链接规则}
  \end{itemize}
\subsubsection{elf format}
\begin{itemize}
\item{elf 头部}
\item{节区头部表}
\item{程序头部表}
\item{.text 节区}
\item{.rodata 节区}
\item{.data 节区}
\item{.bss 节区}
\item{.symtab 节区}
\item{.rel.text 节区}
\item{.rel.data 节区}
\item{.debug 节区}
\end{itemize}
\subsubsection{Loading}
\begin{itemize}
\item{加载器}
  \begin{itemize}
	\item{execve: 程序加载过程}
	\item{程序虚拟地址映像}
	\end{itemize}
\end{itemize}
\subsubsection{Static libraries}
\begin{itemize}
\item{例子}
\item{itemize}
\subsubsection{Shared libraries}
\begin{itemize}
\item{例子}
\end{itemize}
\subsection{Problems}
\subsection{Solutions}
\end{CJK}
\end{document}
