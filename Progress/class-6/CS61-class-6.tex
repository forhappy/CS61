\documentclass[a4paper, 11pt]{article}

%%%%%% use packages %%%%%%
\usepackage{CJKutf8}
\usepackage{graphicx}
\usepackage[unicode]{hyperref}
\usepackage{indentfirst}
\usepackage{latexsym}
\usepackage{amsmath}
\usepackage{flafter}
\usepackage{enumerate}
\usepackage{leftidx}

%%%%%% indent settings %%%%%%
\setlength{\parindent}{20pt}
\setlength{\parskip}{1ex plus 0.5ex minus 0.25ex}
\setlength{\baselineskip}{1.0ex}
\renewcommand{\baselinestretch}{1.5}

\begin{CJK}{UTF8}{gbsn}


\title{CS61第六次课程记录}
\author{傅海平\\
\textsc{Institute Of Computing Technology,}\\
\textsc{Chinese Academy Of Sciences}\\
haipingf@gmail.com\\
}
\date{\today}
\begin{document}
\maketitle
\newpage
\tableofcontents
\newpage
\section{Topics}
\begin{center}
\Large{内存(Memory)及存储(Storage)技术,缓存(Caching)}
\end{center}

\section{Progress}
早上9点开始,9:00 - 10:50 学习 Lec12-Memory\_and\_Storage\_Technologies.pdf 和
Lec13-Caching.pdf 两张课程讲义,然后10:50开始讨论学习过程中遇到的问题
。
\section{Learning Details}
\subsection{Course Sketch}
\subsubsection{Memory}
\begin{itemize}
  \item{静态RAM和动态RAM}
  \item{动态内存常见组织结构,$d \times w$: $d$ supercells of size $w$ bits}

  \item{内存模块}
  \item{加强的DRAM??}
	\begin{itemize}
	  \item{同步DRAM}
	  \item{双倍速同步DRAM}
	  \item{RamBus DRAM}
	  \end{itemize}
	\item{过时的技术}
	  \begin{itemize}
		\item{FPM DRAM}
		\item{EDO DRAM}
		\item{Video RAM}
		\item{CDRAM, GDRAM}
		\end{itemize}
	  \item{非易失性内存ROM, MRAM, FeRAM, PROM, EPROM, EEPROM}
	  \item{传统内存总线}
	  \item{总线容错,Hamming码}
  \end{itemize}
\subsubsection{Disk Drives}
\begin{itemize}
  \item{记录密度}
  \item{道密度}
  \item{面密度}
  \item{寻道时间}
  \item{旋转延迟}
  \item{传输时间}
  \item{\dots}
  \item{逻辑块}
  \end{itemize}
  \subsubsection{IO and Memory Buses}
  \subsubsection{Solid-state Disks}
  \subsubsection{The Principle of Locality}
  \begin{itemize}
	\item{局部性原理:时间局部性和空间局部性}
	\item{时间局部性:Recently referenced memory addresses are likely to be referenced in the near future.}
	\item{空间局部性:Similar memory addresses tend to be referenced close together in time.}
	\end{itemize}
  \subsubsection{Memory Hierarchies}
  \begin{itemize}
	\item{关键点}
	\end{itemize}
  \subsubsection{Caching Concepts}
  \subsubsection{Direct-Mapped Cache Organization}
  \begin{itemize}
	\item{Cache失效:冷失效,容量失效,冲突失效}
	\item{工作集:working set}
	\end{itemize}
  \subsubsection{Set-Associative Cache Organization}
  \subsubsection{Multi-level caches}
  \subsubsection{Cache Writes}
  \begin{itemize}
	\item{写穿策略}
	\item{写回策略}
	\end{itemize}
\subsection{Problems}
\subsection{Solutions}
\end{CJK}
\end{document}
